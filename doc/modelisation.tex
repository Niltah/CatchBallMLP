Nous désirons déplacer le robot en lui fournissant un position de destination absolue $(x, y, z)$ dans un repère lié à la salle. Or, le robot se positionne en utilisant les valeurs angulaires que doivent prendre ses moteurs. Il était donc nécessaire d'implémenter un module permettant de passer d'une position absolue aux valeurs angulaires associées, d'ou la nécessité du calcul d'un modèle géométrique inverse.

\subsection{Modelisation}


